\documentclass[]{article}
\usepackage{lmodern}
\usepackage{amssymb,amsmath}
\usepackage{ifxetex,ifluatex}
\usepackage{fixltx2e} % provides \textsubscript
\ifnum 0\ifxetex 1\fi\ifluatex 1\fi=0 % if pdftex
  \usepackage[T1]{fontenc}
  \usepackage[utf8]{inputenc}
\else % if luatex or xelatex
  \ifxetex
    \usepackage{mathspec}
  \else
    \usepackage{fontspec}
  \fi
  \defaultfontfeatures{Ligatures=TeX,Scale=MatchLowercase}
  \newcommand{\euro}{€}
\fi
% use upquote if available, for straight quotes in verbatim environments
\IfFileExists{upquote.sty}{\usepackage{upquote}}{}
% use microtype if available
\IfFileExists{microtype.sty}{%
\usepackage{microtype}
\UseMicrotypeSet[protrusion]{basicmath} % disable protrusion for tt fonts
}{}
\usepackage[margin=1in]{geometry}
\usepackage{hyperref}
\PassOptionsToPackage{usenames,dvipsnames}{color} % color is loaded by hyperref
\hypersetup{unicode=true,
            pdftitle={TS analysis},
            pdfauthor={J},
            pdfborder={0 0 0},
            breaklinks=true}
\urlstyle{same}  % don't use monospace font for urls
\usepackage{color}
\usepackage{fancyvrb}
\newcommand{\VerbBar}{|}
\newcommand{\VERB}{\Verb[commandchars=\\\{\}]}
\DefineVerbatimEnvironment{Highlighting}{Verbatim}{commandchars=\\\{\}}
% Add ',fontsize=\small' for more characters per line
\usepackage{framed}
\definecolor{shadecolor}{RGB}{248,248,248}
\newenvironment{Shaded}{\begin{snugshade}}{\end{snugshade}}
\newcommand{\KeywordTok}[1]{\textcolor[rgb]{0.13,0.29,0.53}{\textbf{{#1}}}}
\newcommand{\DataTypeTok}[1]{\textcolor[rgb]{0.13,0.29,0.53}{{#1}}}
\newcommand{\DecValTok}[1]{\textcolor[rgb]{0.00,0.00,0.81}{{#1}}}
\newcommand{\BaseNTok}[1]{\textcolor[rgb]{0.00,0.00,0.81}{{#1}}}
\newcommand{\FloatTok}[1]{\textcolor[rgb]{0.00,0.00,0.81}{{#1}}}
\newcommand{\ConstantTok}[1]{\textcolor[rgb]{0.00,0.00,0.00}{{#1}}}
\newcommand{\CharTok}[1]{\textcolor[rgb]{0.31,0.60,0.02}{{#1}}}
\newcommand{\SpecialCharTok}[1]{\textcolor[rgb]{0.00,0.00,0.00}{{#1}}}
\newcommand{\StringTok}[1]{\textcolor[rgb]{0.31,0.60,0.02}{{#1}}}
\newcommand{\VerbatimStringTok}[1]{\textcolor[rgb]{0.31,0.60,0.02}{{#1}}}
\newcommand{\SpecialStringTok}[1]{\textcolor[rgb]{0.31,0.60,0.02}{{#1}}}
\newcommand{\ImportTok}[1]{{#1}}
\newcommand{\CommentTok}[1]{\textcolor[rgb]{0.56,0.35,0.01}{\textit{{#1}}}}
\newcommand{\DocumentationTok}[1]{\textcolor[rgb]{0.56,0.35,0.01}{\textbf{\textit{{#1}}}}}
\newcommand{\AnnotationTok}[1]{\textcolor[rgb]{0.56,0.35,0.01}{\textbf{\textit{{#1}}}}}
\newcommand{\CommentVarTok}[1]{\textcolor[rgb]{0.56,0.35,0.01}{\textbf{\textit{{#1}}}}}
\newcommand{\OtherTok}[1]{\textcolor[rgb]{0.56,0.35,0.01}{{#1}}}
\newcommand{\FunctionTok}[1]{\textcolor[rgb]{0.00,0.00,0.00}{{#1}}}
\newcommand{\VariableTok}[1]{\textcolor[rgb]{0.00,0.00,0.00}{{#1}}}
\newcommand{\ControlFlowTok}[1]{\textcolor[rgb]{0.13,0.29,0.53}{\textbf{{#1}}}}
\newcommand{\OperatorTok}[1]{\textcolor[rgb]{0.81,0.36,0.00}{\textbf{{#1}}}}
\newcommand{\BuiltInTok}[1]{{#1}}
\newcommand{\ExtensionTok}[1]{{#1}}
\newcommand{\PreprocessorTok}[1]{\textcolor[rgb]{0.56,0.35,0.01}{\textit{{#1}}}}
\newcommand{\AttributeTok}[1]{\textcolor[rgb]{0.77,0.63,0.00}{{#1}}}
\newcommand{\RegionMarkerTok}[1]{{#1}}
\newcommand{\InformationTok}[1]{\textcolor[rgb]{0.56,0.35,0.01}{\textbf{\textit{{#1}}}}}
\newcommand{\WarningTok}[1]{\textcolor[rgb]{0.56,0.35,0.01}{\textbf{\textit{{#1}}}}}
\newcommand{\AlertTok}[1]{\textcolor[rgb]{0.94,0.16,0.16}{{#1}}}
\newcommand{\ErrorTok}[1]{\textcolor[rgb]{0.64,0.00,0.00}{\textbf{{#1}}}}
\newcommand{\NormalTok}[1]{{#1}}
\usepackage{graphicx,grffile}
\makeatletter
\def\maxwidth{\ifdim\Gin@nat@width>\linewidth\linewidth\else\Gin@nat@width\fi}
\def\maxheight{\ifdim\Gin@nat@height>\textheight\textheight\else\Gin@nat@height\fi}
\makeatother
% Scale images if necessary, so that they will not overflow the page
% margins by default, and it is still possible to overwrite the defaults
% using explicit options in \includegraphics[width, height, ...]{}
\setkeys{Gin}{width=\maxwidth,height=\maxheight,keepaspectratio}
\setlength{\parindent}{0pt}
\setlength{\parskip}{6pt plus 2pt minus 1pt}
\setlength{\emergencystretch}{3em}  % prevent overfull lines
\providecommand{\tightlist}{%
  \setlength{\itemsep}{0pt}\setlength{\parskip}{0pt}}
\setcounter{secnumdepth}{0}

%%% Use protect on footnotes to avoid problems with footnotes in titles
\let\rmarkdownfootnote\footnote%
\def\footnote{\protect\rmarkdownfootnote}

%%% Change title format to be more compact
\usepackage{titling}

% Create subtitle command for use in maketitle
\newcommand{\subtitle}[1]{
  \posttitle{
    \begin{center}\large#1\end{center}
    }
}

\setlength{\droptitle}{-2em}
  \title{TS analysis}
  \pretitle{\vspace{\droptitle}\centering\huge}
  \posttitle{\par}
  \author{J}
  \preauthor{\centering\large\emph}
  \postauthor{\par}
  \predate{\centering\large\emph}
  \postdate{\par}
  \date{2016年3月31日}


\usepackage{xeCJK}
\setCJKmainfont{simsun.ttc}

% Redefines (sub)paragraphs to behave more like sections
\ifx\paragraph\undefined\else
\let\oldparagraph\paragraph
\renewcommand{\paragraph}[1]{\oldparagraph{#1}\mbox{}}
\fi
\ifx\subparagraph\undefined\else
\let\oldsubparagraph\subparagraph
\renewcommand{\subparagraph}[1]{\oldsubparagraph{#1}\mbox{}}
\fi

\begin{document}
\maketitle

\section{数据导入及预处理}

\begin{Shaded}
\begin{Highlighting}[]
\KeywordTok{library}\NormalTok{(TSA) }\CommentTok{#打开时间序列分析的R包}
\NormalTok{data <-}\StringTok{ }\KeywordTok{read.csv}\NormalTok{(}\StringTok{"shuju.csv"}\NormalTok{) }\CommentTok{#从shuju中读入R环境,命名为data数据框}
\KeywordTok{View}\NormalTok{(data) }\CommentTok{#查看data}
\NormalTok{for(i in }\DecValTok{1}\NormalTok{:}\DecValTok{5}\NormalTok{)\{}
  \KeywordTok{assign}\NormalTok{(}\KeywordTok{paste0}\NormalTok{(}\StringTok{"var"}\NormalTok{,i),}\KeywordTok{log}\NormalTok{(}\KeywordTok{ts}\NormalTok{(data[,i}\DecValTok{+1}\NormalTok{],}\DataTypeTok{star=}\DecValTok{1999}\NormalTok{)))}
\NormalTok{\}}
\CommentTok{#循环赋值,将data数据框中的五个变量的对数转化为ts格式,并进行重命名}
\end{Highlighting}
\end{Shaded}

\section{平稳性分析}

用tseries包的adf.test函数做假设检验,\(H_0\):序列不平稳。

\begin{Shaded}
\begin{Highlighting}[]
\KeywordTok{library}\NormalTok{(tseries)}
\KeywordTok{adf.test}\NormalTok{(var1)}
\end{Highlighting}
\end{Shaded}

\begin{verbatim}
## 
##  Augmented Dickey-Fuller Test
## 
## data:  var1
## Dickey-Fuller = 0.96324, Lag order = 2, p-value = 0.99
## alternative hypothesis: stationary
\end{verbatim}

\begin{Shaded}
\begin{Highlighting}[]
\KeywordTok{adf.test}\NormalTok{(var2)}
\end{Highlighting}
\end{Shaded}

\begin{verbatim}
## 
##  Augmented Dickey-Fuller Test
## 
## data:  var2
## Dickey-Fuller = -2.1699, Lag order = 2, p-value = 0.5077
## alternative hypothesis: stationary
\end{verbatim}

\begin{Shaded}
\begin{Highlighting}[]
\KeywordTok{adf.test}\NormalTok{(var3)}
\end{Highlighting}
\end{Shaded}

\begin{verbatim}
## 
##  Augmented Dickey-Fuller Test
## 
## data:  var3
## Dickey-Fuller = -1.5828, Lag order = 2, p-value = 0.7313
## alternative hypothesis: stationary
\end{verbatim}

\begin{Shaded}
\begin{Highlighting}[]
\KeywordTok{adf.test}\NormalTok{(var4)}
\end{Highlighting}
\end{Shaded}

\begin{verbatim}
## 
##  Augmented Dickey-Fuller Test
## 
## data:  var4
## Dickey-Fuller = -2.7925, Lag order = 2, p-value = 0.2705
## alternative hypothesis: stationary
\end{verbatim}

\begin{Shaded}
\begin{Highlighting}[]
\KeywordTok{adf.test}\NormalTok{(var5)}
\end{Highlighting}
\end{Shaded}

\begin{verbatim}
## 
##  Augmented Dickey-Fuller Test
## 
## data:  var5
## Dickey-Fuller = -2.7228, Lag order = 2, p-value = 0.297
## alternative hypothesis: stationary
\end{verbatim}

大于显著性水平,不能拒绝原假设。

\begin{Shaded}
\begin{Highlighting}[]
\KeywordTok{adf.test}\NormalTok{(}\KeywordTok{diff}\NormalTok{(var1,}\DecValTok{4}\NormalTok{))}
\end{Highlighting}
\end{Shaded}

\begin{verbatim}
## 
##  Augmented Dickey-Fuller Test
## 
## data:  diff(var1, 4)
## Dickey-Fuller = -1.627, Lag order = 2, p-value = 0.7145
## alternative hypothesis: stationary
\end{verbatim}

\begin{Shaded}
\begin{Highlighting}[]
\KeywordTok{adf.test}\NormalTok{(}\KeywordTok{diff}\NormalTok{(var2,}\DecValTok{2}\NormalTok{))}
\end{Highlighting}
\end{Shaded}

\begin{verbatim}
## 
##  Augmented Dickey-Fuller Test
## 
## data:  diff(var2, 2)
## Dickey-Fuller = -1.8582, Lag order = 2, p-value = 0.6264
## alternative hypothesis: stationary
\end{verbatim}

\begin{Shaded}
\begin{Highlighting}[]
\KeywordTok{adf.test}\NormalTok{(}\KeywordTok{diff}\NormalTok{(var3))}
\end{Highlighting}
\end{Shaded}

\begin{verbatim}
## 
##  Augmented Dickey-Fuller Test
## 
## data:  diff(var3)
## Dickey-Fuller = -2.4129, Lag order = 2, p-value = 0.4151
## alternative hypothesis: stationary
\end{verbatim}

\begin{Shaded}
\begin{Highlighting}[]
\KeywordTok{adf.test}\NormalTok{(}\KeywordTok{diff}\NormalTok{(var4))}
\end{Highlighting}
\end{Shaded}

\begin{verbatim}
## 
##  Augmented Dickey-Fuller Test
## 
## data:  diff(var4)
## Dickey-Fuller = -2.5941, Lag order = 2, p-value = 0.346
## alternative hypothesis: stationary
\end{verbatim}

\begin{Shaded}
\begin{Highlighting}[]
\KeywordTok{adf.test}\NormalTok{(}\KeywordTok{diff}\NormalTok{(var5))}
\end{Highlighting}
\end{Shaded}

\begin{verbatim}
## 
##  Augmented Dickey-Fuller Test
## 
## data:  diff(var5)
## Dickey-Fuller = -2.4404, Lag order = 2, p-value = 0.4046
## alternative hypothesis: stationary
\end{verbatim}

\end{document}
